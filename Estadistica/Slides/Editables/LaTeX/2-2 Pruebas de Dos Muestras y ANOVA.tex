
\documentclass[aspectratio=169]{beamer}
\usepackage[utf8]{inputenc}
\usepackage{amsmath}
\usepackage{graphicx}
\usepackage[spanish]{babel}

\usetheme{Madrid}

\usepackage{xcolor}
\definecolor{myblue}{HTML}{427A92}
\setbeamercolor{structure}{fg=myblue}
\setbeamercolor{frametitle}{bg=myblue}

\setbeamerfont{frametitle}{size=\huge}

\title[2.2. Pruebas de Dos Muestras y ANOVA]{\Huge 2.2. Pruebas de Dos Muestras y ANOVA}
\subtitle{2. Estadística y Probabilidad Básica}

\author{Christian E. Galarza}
\date{\bf Programa New Dimensions}

\begin{document}

\frame{\titlepage}

% Diapositiva: Performing t-tests - Introducción
% Diapositiva: Performing t-tests - Introducción
\begin{frame}
    \frametitle{Pruebas $t$ para comparaciones de medias}
    \textbf{¿Qué es una prueba t?}
    \begin{itemize}
        \item Herramienta estadística para comparar las medias de dos grupos.
        \item Determina si las diferencias observadas son por casualidad o estadísticamente significativas.
        \item Esencial en la toma de decisiones basada en datos.
    \end{itemize}

\vfill
    
    \textbf{Ejemplo:} ¿Los empleados con capacitación tienen un rendimiento significativamente diferente que aquellos sin capacitación?
\end{frame}

% Diapositiva: Hypothesis testing workflow
\begin{frame}
    \frametitle{¿Cómo realizar un contraste de hipótesis?}
    \begin{itemize}
        \item Establecer hipótesis nula (\(H_0\)): No hay diferencia entre los dos grupos.
        \item Hipótesis alternativa (\(H_a\)): Hay una diferencia significativa.
        
        \item Calcular estadístico de prueba y el $p$-valor.
        \item Tomar una decisión basada en el $p$-valor.
    \end{itemize}

\end{frame}



\begin{frame}
    \frametitle{Ejemplo: Contraste de hipótesis para empleados con/sin capacitación}
    
    \textbf{1. Establecer hipótesis nula (\(H_0\)) y alternativa (\(H_a\)):}
    \begin{itemize}
        \item \(H_0\): La media del rendimiento de los empleados con capacitación es igual a la de los empleados sin capacitación.
        \item \(H_a\): La media del rendimiento de los empleados con capacitación es diferente de la de los empleados sin capacitación.
    \end{itemize}
    
    \textbf{2. Elegir nivel de significancia (\(\alpha\)):}
    \begin{itemize}
        \item \(\alpha = 0.05\).
    \end{itemize}

\end{frame}

    \begin{frame}
    \frametitle{Ejemplo: Contraste de hipótesis para empleados con/sin capacitación}
    \textbf{3. Calcular estadístico de prueba y el \(p\)-valor:}
    \begin{itemize}
        \item Estadístico de prueba \(t = 2.5\).
        \item \(p\)-valor = \(0.03\).
    \end{itemize}
    
    \textbf{4. Tomar una decisión:}
    \begin{itemize}
        \item \(p\)-valor \(0.03 < \alpha = 0.05\).
        \item Rechazamos \(H_0\): Hay una diferencia significativa en el rendimiento entre los grupos.
    \end{itemize}
\end{frame}





% Diapositiva: Two sample mean test statistic
\begin{frame}
    \frametitle{Estadístico de prueba $t$}
    Fórmula (simplificada):
    \[
    t = \frac{\bar{X}_1 - \bar{X}_2}{\sqrt{\frac{s_1^2}{n_1} + \frac{s_2^2}{n_2}}}
    \]

\vfill

    \textbf{Interpretación:} Cuanto mayor sea el valor de \(t\), mayor será la diferencia entre los grupos.

\vfill
    
    \textbf{Ejemplo:} Comparar las ventas medias de dos tiendas.

\end{frame}

% Diapositiva: Calculating p-values from t-statistics
\begin{frame}
\frametitle{Cálculo de valor $p$ en pruebas $t$}
\begin{center}
\includegraphics[width=0.8\linewidth]{Figuras/how-to-find-p-value.png} % Imagen de una distribución con las colas marcadas.    
\end{center}
\end{frame}


% Diapositiva: Paired t-tests - Introducción
\begin{frame}
    \frametitle{Prueba t pareada}
    \textbf{¿Qué es una prueba t pareada?}
    \begin{itemize}
        \item Compara las medias de dos mediciones relacionadas.
        \item Útil cuando las observaciones son dependientes o emparejadas.
        \item Ejemplo: rendimiento de empleados antes y después de una capacitación.
    \end{itemize}

\end{frame}

% Diapositiva: Is pairing needed?
\begin{frame}
    \frametitle{¿Cuándo necesitamos considerar una prueba pareada?}
    \begin{itemize}
        \item Parear es útil cuando las observaciones no son independientes.
        \item Reduce la variabilidad y mejora la precisión.
        \item Esencial para obtener resultados confiables.
    \end{itemize}

\vfill
    
    \textbf{Otro ejemplo:} Medir la satisfacción del cliente antes y después de implementar un cambio.
\end{frame}

% Diapositiva: Using t.test()
\begin{frame}
    \frametitle{Usando \verb|t.test()|}

    \textbf{t.test()} es una función de R
    \begin{itemize}
        \item Realiza automáticamente pruebas t.
        \item Proporciona estadístico de prueba y p-valor.
        \item Facilita la interpretación y toma de decisiones.
    \end{itemize}

\end{frame}

% Diapositiva: ANOVA tests - Introducción
\begin{frame}
    \frametitle{ANOVA tests}
    \textbf{¿Qué es ANOVA?}
    \begin{itemize}
        \item Herramienta estadística para comparar las medias de tres o más grupos.
        \item Determina si al menos un grupo es diferente de los demás.
        \item Esencial cuando se tienen múltiples categorías o niveles.
    \end{itemize}
\end{frame}



\begin{frame}{Hipótesis para la prueba ANOVA}
\begin{itemize}
    \item \textbf{Hipótesis nula (\(H_0\))}: Todas las medias de los grupos son iguales.
    \item \textbf{Hipótesis alterna (\(H_a\))}: Al menos una de las medias de los grupos es diferente.
\end{itemize}
\end{frame}

\begin{frame}{Hipótesis para la prueba ANOVA}
\begin{itemize}
    \item \textbf{Hipótesis nula (\(H_0\))}: \( \mu_1 = \mu_2 = \ldots = \mu_k \)
    \item \textbf{Hipótesis alterna (\(H_a\))}: Al menos una \( \mu_i \) es diferente de las otras.
\end{itemize}
\end{frame}




% Diapositiva: Calculating p-values from t-statistics
\begin{frame}
\frametitle{Prueba ANOVA}
\begin{center}
\includegraphics[width=0.8\linewidth]{Figuras/tabla-anova-spss.jpg} % Imagen de una distribución con las colas marcadas.    
\end{center}
\end{frame}




% Diapositiva: Conducting an ANOVA test
\begin{frame}
    \frametitle{Prueba ANOVA}
    \begin{itemize}
        \item Establecer hipótesis.
        \item Calcular estadístico F y p-valor.
        \item Si p-valor < \(\alpha\), al menos un grupo es diferente.
        \item Interpretar resultados y tomar decisiones.
    \end{itemize}

\vfill

    
    \textbf{Ejemplo:} Comparar las ventas en diferentes regiones.
\end{frame}




% Diapositiva: Pairwise t-tests
\begin{frame}
    \frametitle{Pruebas $t$ para todos los pares}
    \textbf{¿Por qué pruebas t por pares después de ANOVA?}
    \begin{itemize}
        \item ANOVA nos dice si hay diferencias, pero no entre qué grupos.
        \item Las pruebas t por pares comparan cada par de grupos.
        \item Ayuda a identificar dónde están las diferencias.
    \end{itemize}

\vfill
    \textbf{Ejemplo:} Determinar qué regiones tienen ventas significativamente diferentes.
\end{frame}



\end{document}
