\documentclass[aspectratio=169]{beamer}
\usepackage[utf8]{inputenc}
\usepackage{amsmath}
\usepackage{graphicx}
\usepackage[spanish]{babel}

\usetheme{Madrid}
\usepackage{xcolor}
\definecolor{myblue}{HTML}{427A92}
\setbeamercolor{structure}{fg=myblue}
\setbeamercolor{frametitle}{bg=myblue}
\setbeamerfont{frametitle}{size=\huge}

\title[1.4. Correlación y Diseño Experimental]{\Huge 1.4 Correlación y Diseño Experimental}
\subtitle{1. Estadística y Probabilidad Básica}
\author{Christian E. Galarza}
\date{\bf Programa New Dimensions}

\begin{document}

\frame{\titlepage}

\section{Correlación y Diseño Experimental}

\subsection{Correlación}
\begin{frame}
\frametitle{Correlación}
\begin{itemize}
    \item La correlación mide la fuerza y dirección de la relación lineal entre dos variables.
    \item El coeficiente de correlación de Pearson varía entre -1 y 1.
    \item \( r = 1 \): correlación positiva perfecta; \( r = -1 \): correlación negativa perfecta; \( r = 0 \): no hay correlación.
\end{itemize}
\end{frame}

\begin{frame}
\frametitle{Caveats de la Correlación}
\begin{itemize}
    \item La correlación no implica causalidad.
    \item Sensible a valores atípicos.
    \item No puede medir relaciones no lineales.
\end{itemize}
\end{frame}

\begin{frame}
\frametitle{Transformación de Variables}
\begin{itemize}
    \item A veces, la transformación de variables puede revelar una relación lineal.
    \item Ejemplo: aplicar logaritmo o raíz cuadrada.
\end{itemize}
\end{frame}

\subsection{Confounders y Diseño Experimental}
\begin{frame}
\frametitle{Confounders}
\begin{itemize}
    \item Un confounder es una variable que está relacionada tanto con la variable independiente como con la dependiente.
    \item Puede crear o ocultar una relación aparente entre las dos variables de interés.
\end{itemize}
\end{frame}

\subsection{Introducción al Diseño de Experimentos}
\begin{frame}
\frametitle{¿Qué es el Diseño de Experimentos?}
\begin{itemize}
    \item Es una técnica estadística utilizada para planificar, diseñar y analizar experimentos.
    \item Ayuda a maximizar la obtención de información, minimizando el uso de recursos.
    \item Se enfoca en identificar las causas de variación en procesos y sistemas.
    \item Permite estudiar los efectos de una o más variables independientes sobre una variable respuesta.
\end{itemize}
\end{frame}

\begin{frame}
\frametitle{Principios Básicos}
\begin{itemize}
    \item \textbf{Replicación:} Realizar el mismo experimento varias veces para obtener una estimación de la variabilidad.
    \item \textbf{Randomización:} Asignar al azar las unidades experimentales para evitar sesgos.
    \item \textbf{Bloqueo:} Agrupar unidades experimentales similares para reducir la variabilidad y aislar factores de interés.
\end{itemize}
Ejemplo: Si queremos evaluar un nuevo medicamento, las personas podrían ser asignadas al azar a grupos de tratamiento o control, y se podría replicar el estudio en diferentes lugares para obtener resultados más robustos.
\end{frame}


\begin{frame}
\frametitle{Tipos de Estudios}
\begin{itemize}
    \item Estudios Observacionales: el investigador observa sin intervenir.
    \item Estudios Experimentales: el investigador interviene y controla las variables.
\end{itemize}
\end{frame}

% Diapositiva sobre Estudios Observacionales
\begin{frame}
\frametitle{Estudios Observacionales}
\begin{itemize}
    \item Se basan en la observación y registro de eventos o características.
    \item No hay intervención por parte del investigador.
    \item Pueden ser descriptivos (simplemente observan y describen) o analíticos (buscan relaciones entre variables).
    \item Ejemplos: Estudios de cohortes, estudios transversales y estudios de casos y controles.
\end{itemize}
\end{frame}

% Diapositiva sobre Estudios Experimentales
\begin{frame}
\frametitle{Estudios Experimentales}
\begin{itemize}
    \item El investigador interviene y controla una o más variables.
    \item Se busca determinar una relación causa-efecto.
    \item Suelen incluir un grupo control y un grupo experimental.
    \item Los participantes son asignados aleatoriamente a los grupos.
    \item Ejemplo: Ensayos clínicos controlados, donde se evalúa la efectividad de un tratamiento.
\end{itemize}
\end{frame}


\begin{frame}
\frametitle{Estudios Longitudinales vs. Transversales}
\begin{itemize}
    \item Estudios Longitudinales: se estudian a los individuos o unidades durante un período de tiempo.
    \item Estudios Transversales: observan a los individuos o unidades en un solo punto en el tiempo.
\end{itemize}
\end{frame}


% Diapositiva sobre Estudios Longitudinales
\begin{frame}
\frametitle{Estudios Longitudinales}
\begin{itemize}
    \item Estudian a los mismos sujetos a lo largo de un período de tiempo.
    \item Observan cambios o evoluciones en las características o variables de interés.
    \item Pueden durar desde meses hasta décadas.
    \item Son útiles para identificar tendencias o cambios a lo largo del tiempo.
    \item Ejemplo: Un estudio que sigue a una cohorte de individuos desde su infancia hasta la adultez para estudiar el desarrollo de ciertas enfermedades.
\end{itemize}
\end{frame}

% Diapositiva sobre Estudios Transversales
\begin{frame}
\frametitle{Estudios Transversales}
\begin{itemize}
    \item Se realizan en un punto específico en el tiempo.
    \item Se observan diferentes grupos o categorías al mismo tiempo.
    \item No se sigue a los mismos sujetos a lo largo del tiempo como en los estudios longitudinales.
    \item Son útiles para obtener una ``fotografía'' instantánea de una población en un momento dado.
    \item Ejemplo: Una encuesta de salud que busca determinar la prevalencia de una enfermedad en diferentes grupos de edad en un año específico.
\end{itemize}
\end{frame}



% \subsection{Ejemplo Práctico}
% \begin{frame}
% \frametitle{Ejemplo: Correlación entre Azúcar y Felicidad}
% Supongamos que queremos estudiar la relación entre la cantidad de azúcar consumida y el nivel de felicidad en un grupo de individuos. Tenemos los siguientes datos:

% \begin{itemize}
%     \item Azúcar consumida (en gramos): \( X = [30, 40, 50, 60, 70] \)
%     \item Nivel de felicidad (en una escala de 1 a 10): \( Y = [3, 4, 5, 6, 7] \)
% \end{itemize}

% Queremos calcular la correlación entre estas dos variables y determinar si hay una relación lineal positiva.
% \end{frame}

% \begin{frame}[fragile]
% \frametitle{Cálculo de la Correlación en R}
% Podemos calcular la correlación en R usando la función \texttt{cor()}.

% \begin{verbatim}
% # Datos
% azucar <- c(30, 40, 50, 60, 70)
% felicidad <- c(3, 4, 5, 6, 7)

% # Calcular la correlación
% correlacion <- cor(azucar, felicidad)
% \end{verbatim}

% Resultado: La correlación entre la cantidad de azúcar consumida y el nivel de felicidad es aproximadamente 1, lo que indica una fuerte relación lineal positiva.
% \end{frame}


\subsection{Ejemplo Práctico}
\begin{frame}
\frametitle{Ejemplo: Inversión en Publicidad vs Ventas}
Supongamos que una empresa quiere analizar la relación entre la inversión en publicidad (en miles de dólares) y las ventas mensuales de un producto (en miles de unidades). Se recopilan los datos de los últimos 5 meses:

\begin{itemize}
    \item Inversión en publicidad: \( X = [10, 20, 30, 40, 50] \)
    \item Ventas mensuales: \( Y = [50, 55, 65, 75, 90] \)
\end{itemize}

El objetivo es determinar si invertir más en publicidad está correlacionado con un aumento en las ventas.
\end{frame}

\begin{frame}[fragile]
\frametitle{Cálculo de la Correlación en R}
Para evaluar esta relación, podemos calcular la correlación en R utilizando la función \texttt{cor()}.

\begin{verbatim}
# Datos
inversion <- c(10, 20, 30, 40, 50)
ventas <- c(50, 55, 65, 75, 90)

# Calcular la correlación
correlacion <- cor(inversion, ventas)
\end{verbatim}

\end{frame}


\end{document}
