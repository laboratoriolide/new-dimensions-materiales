\documentclass[aspectratio=169]{beamer}
\usepackage[utf8]{inputenc}
\usepackage{amsmath}
\usepackage{graphicx}
\usepackage[spanish]{babel}

\usetheme{Madrid}

\usepackage{xcolor}
\definecolor{myblue}{HTML}{427A92}
\setbeamercolor{structure}{fg=myblue}
\setbeamercolor{frametitle}{bg=myblue}

\setbeamerfont{frametitle}{size=\huge}

\title[2.4. Pruebas No Paramétricas]{\Huge 2.4. Pruebas No Paramétricas}
\subtitle{2. Estadística y Probabilidad Básica}

\author{Christian E. Galarza}
\date{\bf Programa New Dimensions}


\begin{document}

\frame{\titlepage}

% Diapositiva 1: Introducción a las Pruebas No Paramétricas
\begin{frame}
\frametitle{Introducción a las Pruebas No Paramétricas}
\begin{itemize}
\item Alternativa a las pruebas paramétricas.
\item No asumen una distribución específica.
\item Útiles cuando no se cumplen supuestos.
\end{itemize}
\end{frame}

% Diapositiva 3: Supuestos Comunes en Pruebas de Hipótesis
\begin{frame}
\frametitle{Supuestos Comunes en Pruebas de Hipótesis}
\begin{itemize}
\item Independencia de observaciones.
\item Normalidad de los datos.
\item Homocedasticidad.
\end{itemize}
\end{frame}


\begin{frame}
\frametitle{Supuestos en Pruebas de Hipótesis}

\begin{itemize}
    \item \textbf{T de Student (Una media):}
    \begin{itemize}
        \item Muestra aleatoria.
        \item Distribución normal o tamaño de muestra grande.
        \item Varianza conocida o desconocida pero estimable.
    \end{itemize}

\vfill

    \item \textbf{T de Student (Pares de medias):}
    \begin{itemize}
        \item Muestras aleatorias y dependientes.
        \item Diferencias distribuidas normalmente.
    \end{itemize}
        \end{itemize}


\end{frame}

\begin{frame}
\frametitle{Supuestos en Pruebas de Hipótesis}
    \begin{itemize}
    \item \textbf{1 y 2 proporciones:}
    \begin{itemize}
        \item Muestras aleatorias e independientes.
        \item Tamaño de muestra suficientemente grande (para que np y n(1-p) sean $\geq$ 5).
    \end{itemize}

\vfill

    \item \textbf{ANOVA:}
    \begin{itemize}
        \item Muestras aleatorias e independientes.
        \item Distribuciones normalmente distribuidas en cada grupo.
        \item Homocedasticidad (varianzas iguales entre grupos).
    \end{itemize}
\end{itemize}

\end{frame}





% Diapositiva 4: El Tamaño de la Muestra
\begin{frame}
\frametitle{El Tamaño de la Muestra}
\begin{itemize}
    \item Determina poder estadístico.
    \item Afecta precisión y confiabilidad.
    \item Implicaciones en validez de pruebas.
    \item Si la muestra no es lo suficientemente grande para pruebas de proporción o medias:
    \begin{itemize}
        \item Supuesto de normalidad puede omitirse...
        \item ... si se cumple el Teorema del Límite Central.
    \end{itemize}
\end{itemize}
\end{frame}


% Diapositiva 5: Generación de Datos y Estadísticas
\begin{frame}
\frametitle{Generación de Datos y Estadísticas}
\begin{itemize}
    \item \textbf{Uso de simulaciones:}
    \begin{itemize}
        \item Generar muestras aleatorias bajo ciertas hipótesis.
        \item Estimar la distribución de una estadística.
        \item Comprender escenarios bajo diferentes supuestos.
    \end{itemize}
    
    \item \textbf{Herramientas estadísticas:}
    \begin{itemize}
        \item Softwares como R, Python, SAS y SPSS.
        \item Funciones y paquetes específicos para pruebas no paramétricas.
    \end{itemize}
    
    \item \textbf{Técnicas para pruebas no paramétricas:}
    \begin{itemize}
        \item Uso de rangos en lugar de datos brutos.
        \item Pruebas como Wilcoxon, Kruskal-Wallis.
        \item Aplicación cuando no se cumplen supuestos de normalidad o varianza homogénea.
    \end{itemize}
\end{itemize}
\end{frame}

% Diapositiva 6: Estadística Observada y Valor-p
\begin{frame}
\frametitle{Estadística Observada y Valor-p}
\[ H_0: \mu = \mu_0 \]
\[ H_1: \mu \neq \mu_0 \]
\begin{itemize}
    \item \textbf{Interpretación del valor-p:}
    \begin{itemize}
        \item Probabilidad bajo la hipótesis nula.
        \item Un valor-p pequeño sugiere evidencia contra \(H_0\).
    \end{itemize}
    
    \item \textbf{Decisión basada en nivel de significancia:}
    \begin{itemize}
        \item Usualmente \( \alpha = 0.05 \).
        \item Si valor-p \(< \alpha\), rechazamos \(H_0\).
        \item Significa que los datos observados son incompatibles con \(H_0\) a ese nivel de significancia.
    \end{itemize}
\end{itemize}
\end{frame}


% Diapositiva 7a: Matemática detrás del ANOVA No Paramétrico
\begin{frame}
\frametitle{Matemática de ANOVA No Paramétrico}
\begin{itemize}
\item Kruskal-Wallis: versión no paramétrica de ANOVA.
\item Estadístico \( H \): basado en rangos.
\item Distribución aproximadamente \(\chi^2\) con \( k - 1 \) grados de libertad.
\end{itemize}
\end{frame}

% Diapositiva 1: Proceso de cálculo de rangos - Paso 1 y 2
\begin{frame}
\frametitle{Cálculo de Rangos - Pasos 1 y 2}
\textbf{1. Ordenar los Datos}
\begin{itemize}
\item Ordena todos los datos, sin importar el grupo.
\item De menor a mayor.
\end{itemize}

\textbf{2. Asignar Rangos}
\begin{itemize}
\item Asigna un rango al dato más pequeño (rango 1).


\item Continúa consecutivamente: el siguiente dato recibe el rango 2, y así sucesivamente.
\end{itemize}
\end{frame}

% Diapositiva 2: Proceso de cálculo de rangos - Paso 3
\begin{frame}
\frametitle{Cálculo de Rangos - Paso 3}
\textbf{3. Tratamiento de Empates (Datos Repetidos)}
\begin{itemize}
\item Si dos o más datos son idénticos, se les da el promedio de los rangos que habrían obtenido si fueran diferentes.
\item Ejemplo: Si los rangos 3 y 4 corresponden a datos idénticos, ambos datos reciben el rango \( \frac{3 + 4}{2} = 3.5 \).
\end{itemize}
\vfill

\textbf{Ejemplo:}\\ 
Datos: 5, 5, 6, 7, 7, 7\\
Rangos: 1.5, 1.5, 3, 5, 5, 5.
\end{frame}



% Diapositiva 7b: Ejemplo de ANOVA No Paramétrico
\begin{frame}
\frametitle{Ejemplo de ANOVA No Paramétrico}
\textbf{Enunciado:} Suponga que se quiere comparar la satisfacción de clientes entre tres tiendas distintas. Las valoraciones no siguen una distribución normal. ¿Existen diferencias significativas entre las tiendas?
\end{frame}



% Diapositiva: Ejemplo de cálculo de rangos y Prueba de Kruskal-Wallis
\begin{frame}
\frametitle{Prueba de Kruskal-Wallis y cálculo de rangos}
\textbf{Datos (Satisfacción de clientes en 3 tiendas):}
\begin{align*}
\text{Tienda A:} & \quad 5, 6, 7 \\
\text{Tienda B:} & \quad 8, 9, 6 \\
\text{Tienda C:} & \quad 7, 7, 8 
\end{align*}
\textbf{Rangos:}
\begin{align*}
\text{5:} & \quad 1 \\
\text{6:} & \quad 2.5 \quad (\text{promedio de rangos para 6 y 6}) \\
\text{7:} & \quad 5 \quad (\text{promedio de rangos para 7, 7 y 7}) \\
\text{8:} & \quad 7.5 \quad (\text{promedio de rangos para 8 y 8}) \\
\text{9:} & \quad 9 \\
\end{align*}

\textbf{Estadístico Kruskal-Wallis \( H \):} Se calcula usando los rangos asignados y las fórmulas correspondientes. 

\textbf{Resultado:} Si el valor \( H \) es lo suficientemente grande, rechazamos la hipótesis nula de que las medianas de las tiendas son iguales.

\end{frame}


% Diapositiva 8a: Matemática detrás de Prueba t basada en Simulación
\begin{frame}
\frametitle{Prueba t basada en Simulación}
\begin{itemize}
\item Estadístico \( t \): mismo que en prueba t tradicional.
\item En lugar de asumir una distribución t de Student, generamos una distribución empírica simulando bajo la hipótesis nula.
\item El p-valor se obtiene comparando el estadístico observado con la distribución simulada.
\end{itemize}
\end{frame}

% Diapositiva 8b: Pasos para realizar Prueba t basada en Simulación
\begin{frame}
\frametitle{Prueba t basada en Simulación (*)}
\begin{enumerate}
\item Calcular el estadístico \( t \) para las muestras observadas.
\item Combinar ambas muestras y redistribuirlas aleatoriamente en dos nuevos grupos (simulación bajo H0).
\item Calcular el estadístico \( t \) para esta redistribución.
\item Repetir paso 2 y 3 muchas veces (ejemplo, 10,000 veces) para construir una distribución de \( t \) bajo la hipótesis nula.
\item El p-valor es la proporción de veces que el \( t \) simulado es tan extremo o más extremo que el observado.
\end{enumerate}
\end{frame}


% Diapositiva 9a: Generalidades de Prueba de Suma de Rangos
\begin{frame}
\frametitle{Prueba de Suma de Rangos}
\begin{itemize}
\item Es una prueba no paramétrica.
\item Utilizada para comparar las {\bf medianas entre dos grupos}.
\item Aplicada cuando se tienen datos ordinales o rangos.
\item No asume normalidad de los datos.
\item Prueba más utilizada: Test de Wilcoxon.
\end{itemize}
\end{frame}

% Diapositiva 9b: Formulación Matemática de la Prueba de Suma de Rangos
\begin{frame}
\frametitle{Prueba de Suma de Rangos}
\begin{itemize}
\item Se asignan rangos a cada dato combinando ambos grupos.
\item \( R_1 \): suma de rangos del grupo 1.
\item \( R_2 \): suma de rangos del grupo 2.
\item Estadístico de test \( W \) se basa en \( R_1 \) o \( R_2 \) (e.g., \( W = R_1 \) para Wilcoxon).
\item Se compara \( W \) con una distribución esperada bajo H0.
\end{itemize}
\end{frame}

% Diapositiva 9c: Observaciones y Ejemplo de Prueba de Suma de Rangos
\begin{frame}
\frametitle{Prueba de Suma de Rangos}
\begin{itemize}
\item Observaciones:
  \begin{itemize}
  \item Útil cuando los datos tienen outliers.
  \item Menos poderosa que pruebas paramétricas con datos normales.
  \end{itemize}
\item Ejemplo:
  \begin{itemize}
  \item Queremos saber si dos técnicas de enseñanza tienen diferencias significativas en el rendimiento de estudiantes.
  \item Los datos se presentan como rankings de rendimiento y no cumplen con normalidad.
  \item Usamos la prueba de suma de rangos para comparar las medianas de rendimiento de ambos grupos.
  \end{itemize}
\end{itemize}
\end{frame}


\begin{frame}
\frametitle{Ejercicio Práctico: Prueba de Wilcoxon para Comparación de Medias}
\textbf{Contexto:} 
Una empresa de ventas al por menor quiere comparar las ventas antes y después de una campaña publicitaria. Selecciona aleatoriamente a 10 tiendas y registra sus ventas semanales antes y después de la campaña. ¿Tuvo un impacto significativo la campaña publicitaria?

\end{frame}


\begin{frame}
\frametitle{Ejercicio Práctico: Prueba de Wilcoxon para Comparación de Medias}
\textbf{Datos:}
\begin{table}
\begin{tabular}{|c|c|c|}
\hline
Tienda & Ventas Antes (miles) & Ventas Después (miles) \\
\hline
1 & 50 & 55 \\
2 & 45 & 47 \\
3 & 60 & 62 \\
4 & 53 & 54 \\
5 & 49 & 52 \\
6 & 52 & 58 \\
7 & 48 & 49 \\
8 & 55 & 57 \\
9 & 54 & 56 \\
10 & 50 & 51 \\
\hline
\end{tabular}
\end{table}
¿Las ventas mejoraron significativamente después de la campaña?
\end{frame}



\begin{frame}
\frametitle{Desarrollo Matemático: Prueba de Wilcoxon}
Se desea probar:
\[ H_0: \text{Las medianas de las ventas antes y después son iguales} \]
\[ H_1: \text{Las medianas de las ventas después son mayores que antes} \]

Pasos:
\begin{enumerate}
\item Calcular las diferencias (Después - Antes).
\item Ordenar las diferencias en valor absoluto.
\item Asignar rangos a las diferencias. En caso de empates, tomar el promedio de los rangos.
\item Sumar los rangos de las diferencias positivas (R+).
\item Utilizar la estadística de Wilcoxon: \( W = min(R+, R-) \).
\item Comparar \( W \) con la distribución de Wilcoxon para determinar significancia.
\end{enumerate}
\end{frame}

\begin{frame}[fragile]
\begin{verbatim}
# Datos
ventas_antes <- c(50, 45, 60, 53, 49, 52, 48, 55, 54, 50)
ventas_despues <- c(55, 47, 62, 54, 52, 58, 49, 57, 56, 51)

# Realizar la prueba de Wilcoxon
resultado <- wilcox.test(ventas_antes, ventas_despues, paired = TRUE, alternative = "less")

# Imprimir p-valor
print(resultado$p.value)
\end{verbatim}
Si el \( p-valor \) es menor que 0.05, rechazamos \( H_0 \) y concluimos que las ventas mejoraron significativamente después de la campaña.
\end{frame}




% Diapositiva 10: Conclusión y Recomendaciones
\begin{frame}
\frametitle{Conclusión y Recomendaciones}
\begin{itemize}
\item \textbf{Versatilidad}: Las técnicas no paramétricas son esenciales cuando los datos no cumplen con supuestos específicos.
\item \textbf{Investigación Robusta}: Permite la correcta interpretación de resultados, especialmente con datos no normales o con outliers.
\item \textbf{Relevancia en el Mundo Real}: Muchos conjuntos de datos en la vida real no cumplen con los supuestos de normalidad.
\end{itemize}
\end{frame}



\begin{frame}
\frametitle{¡Atención!}
\begin{itemize}
    \item \textbf{Próxima clase:}
    \begin{itemize}
        \item Quiz de preguntas básicas sobre la Unidad 2.
    \end{itemize}
    \item \textbf{Práctica:}
    \begin{itemize}
        \item Ejercicios prácticos en R.
    \end{itemize}
    \item \textbf{Tareas:}
    \begin{itemize}
        \item Se enviará un deber relacionado con la Unidad 1 y 2.
    \end{itemize}
\end{itemize}
\end{frame}



\end{document}




\begin{document}

\frame{\titlepage}

\section{Pruebas No Paramétricas}

\begin{frame}
\frametitle{Supuestos en las Pruebas de Hipótesis}

\begin{itemize}
    \item Las pruebas paramétricas requieren supuestos sobre la distribución de los datos.
    \item Las pruebas no paramétricas no requieren estos supuestos.
    \item Ejemplos de supuestos: normalidad, homogeneidad de varianzas.
\end{itemize}

\end{frame}

\begin{frame}
\frametitle{Pruebas de Tamaño de Muestra}

\begin{itemize}
    \item Importante para determinar si la muestra es lo suficientemente grande.
    \item Puede afectar la validez de los resultados.
\end{itemize}

\end{frame}

\begin{frame}
\frametitle{El Marco "Solo Hay Una Prueba"}

\begin{itemize}
    \item Un enfoque unificado para realizar pruebas de hipótesis.
    \item Incluye especificar, hipotetizar, generar y calcular.
\end{itemize}

\end{frame}

\begin{frame}
\frametitle{ANOVA y Pruebas t No Paramétricas}

\begin{itemize}
    \item Alternativas a las pruebas paramétricas cuando no se cumplen los supuestos.
    \item Ejemplos: Prueba de suma de rangos, prueba t basada en simulación.
\end{itemize}

\end{frame}

\begin{frame}
\frametitle{Ejemplo Práctico: Prueba de Suma de Rangos}

Supongamos que queremos comparar dos grupos con distribuciones desconocidas.

\begin{itemize}
    \item Grupo A: 10, 20, 15, 12, 18
    \item Grupo B: 5, 7, 8, 6, 9
\end{itemize}

Utilizaremos la prueba de suma de rangos para determinar si hay una diferencia significativa.

\end{frame}

\begin{frame}[fragile]
\frametitle{Código en R: Prueba de Suma de Rangos}

\begin{verbatim}
# Datos
grupo_a <- c(10, 20, 15, 12, 18)
grupo_b <- c(5, 7, 8, 6, 9)

# Realizar la prueba de suma de rangos
resultado <- wilcox.test(grupo_a, grupo_b)
print(resultado)
\end{verbatim}

\end{frame}

\begin{frame}
\frametitle{Interpretación del Resultado}

Si el valor p obtenido es menor que el nivel de significancia (por ejemplo, 0.05), rechazamos la hipótesis nula y concluimos que hay una diferencia significativa entre los grupos.

\end{frame}


\end{document}
